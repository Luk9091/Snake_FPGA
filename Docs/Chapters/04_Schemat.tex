\section{Opis całego układu}
    \subsection{Schemat blokowy}
        \begin{figure}[!ht]
            \centering
            \begin{circuitikz}
                \draw
                    (0,   0) node[draw, minimum width = 4cm, minimum height = 2cm](clk){Podział zegarów}
                    (6,   0) node[draw, minimum width = 4cm, minimum height = 2cm](led){Animacja LED}
                    (8,  -3) node[draw, minimum width = 4cm, minimum height = 2cm](vga){VGA}
                    (4,  -7) node[draw, minimum width = 4cm, minimum height = 2cm](map){Kontroler mapy}
                    (12, -7) node[draw, minimum width = 4cm, minimum height = 2cm](score){Wyświetlanie punktów}
                    (7, -10) node[draw, minimum width = 4cm, minimum height = 2cm](snake){Wąż}
                    (0, -7) node[draw, minimum width = 2cm, minimum height = 2cm, align=center](point){Generator\\punkow}
                    (4, -14) node[draw, minimum width = 4cm, minimum height = 2cm, align=center](mem)  {Pamięć pozycji  \\ (ROM)}
                    (10,-14) node[draw, minimum width = 4cm, minimum height = 2cm, align=center](fifo) {Pamięć kierunku \\ (FIFO)}
                    (10.5, -9.5)node[and port, anchor=in 1, rotate = 90](score_en){}

                    (led.east) to[short, -o, i=\ ] ++ (4, 0) node[right]{Leds}
                    (0, -10) coordinate(turn) node[left, align=center]{Turn\\left/right} (turn) to[short, i=\ , o-] (snake.west)

                    (clk.south) to[short] ++ (0, -2) to[short, i= 25MHz] (vga.west)
                    (clk.east) to[short, i=8Hz] (led.west)
                    % (clk.south) to[short] ++ (0, -6) to[short, i=100MHz, *-] (map.west)
                    % (clk.south) to[short] ++ (0, -6) to[short, i=100MHz] ++ (0, -2)

                    (map.south) to[short, i=XY] ++ (0, -1.25) -- ++ (1, 0)

                    (point.east) to[short, i=XY] (map.west)
                    (snake.north) ++ (-1.5, 0) to[short, i_=Draw] ++ (0, 1) 
                    (snake.north) ++ (0, 0) to[short] ++ (0, 1.5) to[short, i<_=Push] ++(-1, 0)
                    (snake.north) ++ (1, 0) to[short] ++ (0, 2.5) to[short, i<_=Destroy] ++(-2, 0)
                    (snake.east) ++ (0, 0.5) to[short, i>_=Destroy] (score_en.in 1)
                    (fifo.north) ++ (0.9, 0) to[short, i_=Score] (score_en.in 2)

                    (vga) to[short, -*, i=XY] ++ (0, -2.5) coordinate(xy)
                    (xy) -| (score.north)
                    (xy) -| (map.north)

                    (score.north) ++ (1, 0) to[short] ++ (0, 1) to[short, i_=Score] ++ (-3.5, 0) --++(0, 1)
                    (map.north) ++ (-1, 0) to[short] ++ (0, 1) to[short, i=Color] ++ (3.5, 0) --++(0, 1)

                    (snake.west) ++ (0, -0.75) to[short] ++ (-1, 0) coordinate(xy) to[short, i_=FILL] ++ (0, -2.25)
                    % (xy) to[short, *-, i=\ ] ++(0, 2)
                    (snake.south) ++ (-1.75, 0) to[short, i_=RW] ++ (0, -2)
                    (mem.east) -- ++ (0.5, 0) to[short, l=XY] ++ (0, 3)

                    (fifo.west)-- ++ (-.5, 0) to[short, a=Dir] ++ (0, 3)
                    (snake.east) ++ (0, -.5) --++ (0.5, 0) to[short, i=RW] ++ (0, -2.5)


                    (vga.east) ++ (0, 0.5) to[short, i=\ , -o] ++ (2, 0) node[right]{$H_{sync}$}
                    (vga.east) ++ (0, 0.0) to[short, i=\ , -o] ++ (2, 0) node[right]{$V_{sync}$}
                    (vga.east) ++ (0,-0.5) to[short, i=\ , -o] ++ (2, 0) node[right]{$RGB$}
                ;
            \end{circuitikz}
            \renewcommand{\figurename}{Schemat}
            \caption{Schemat blokowy całej gry}
            \label{schematic:all_game}
        \end{figure}

        Na schemacie \ref{schematic:all_game}, przedstawiono uproszczony widok całego układu.
        Do każdego z elementów do którego nie został dociągnięty żaden sygnał zegarowy domyślnie jest podłączony zegar główny $100MHz$.
        Dodatkowo pamięci (pamięć pozycji oraz pamięć kierunku), zostały stworzone jako IP Core.

        % \subsubsection{Opis wyprowadzeń}
        \begin{table}[!ht]
            \centering
            \begin{tabular}{c | c}
                Wejścia & Wyjścia\\\hline
                & Leds[8]\\ 
                Turn left/right[2]& $H_{sync}$, $V_{sync}$ \\
                & $RGB$[3]

            \end{tabular}
            \caption{Opis wyprowadzeń}
        \end{table}
    % \subsection{Opis słowny}
    \subsection{Tabela wyprowadzeń}
    \begin{table}[!ht]
        \begin{tabularx}{\textwidth}{|l|c|c|X|}\hline
            Nr. & Nazwa & kierunek & Opis\\\hline
            \hline
            \multicolumn{4}{|c|}{\textbf{Top}}\\\hline
            1. & CLK                & in & wejście zegara modułu $100MHz$\\\hline
            2. & Reset              & in & wejście głównego resetu -- przycisk pop prawej stronie modułu\\\hline
            3. & SoftReset          & in & wejście resetu gry  -- przycisk pop lewej  stronie modułu\\\hline
            4. & LED $\#8$          & out & wyprowadzenia do diod LED\\\hline
            5. & H\_sync            & out & wyjście synchronizacji poziomej\\\hline
            6. & V\_sync            & out & wyjście synchronizacji pionowej\\\hline
            7. & RGB $\#3$          & out & wyjście koloru\\\hline
            8. & leftRight $\#2$    & in & wejścia do przycisków sterujących\\\hline
            \hline
            \multicolumn{4}{|c|}{\textbf{Top $\rightarrow$ VGA}}\\\hline
            1. & CLK    & in  & wejście zegara taktującego VGA $25MHz$\\\hline
            2. & Reset  & in  & wejście resetu - połączone bezpośrednio do resetu głównego\\\hline
            3. & V\_sync& out & wyjście synchronizacji poziomej generowanej w układzie \\\hline
            4. & H\_sync& out & wyjście synchronizacji pionowej generowanej w układzie \\\hline
            5. & RGB   $\#3$& out & wyjście koloru, będące iloczynem logicznym wejścia \textit{Color} i zezwolenia na wyświetlanie\\\hline
            6. & Color $\#3$& in  & wejście koloru, podawane przez mapę\\\hline
            7. & x $\#10$& out & wyjście wewnątrz układowe aktualnie wyświetlanej współrzędnej $x$ \\\hline
            8. & y $\#9$& out & wyjście wewnątrz układowe aktualnie wyświetlanej współrzędnej $y$ \\\hline
            \hline
            \multicolumn{4}{|c|}{\textbf{Top $\rightarrow$ Map\_controler}}\\\hline
            1. & CLK         & in  & wejście głównego zegara $100MHz$ \\\hline
            2. & Reset       & in  & logiczny iloczyn zegara głównego i zegara gry\\\hline
            3. & x $\#7$     & in  & główna współrzędna wyświetlanej kratki\\\hline
            4. & sub\_x $\#4$& in  & $\frac{1}{16}$ współrzędnej $x$ do wyświetlania sprite'ów (nie używane)\\\hline
            5. & y $\#6$     & in  & główna współrzędna wyświetlanej kratki\\\hline
            6. & sub\_y $\#4$& in  & $\frac{1}{16}$ współrzędnej $y$ do wyświetlania sprite'ów (nie używane)\\\hline
            7. & color $\#3$ & out & wyjście wyświetlanego koloru w odpowiedzi na podane współrzędne\\\hline
            8. & leftRight$\#3$& in  & wejście sterujące wężem \\\hline
            \hline
            \multicolumn{4}{|c|}{\textbf{Top $\rightarrow$ Map\_controler $\rightarrow$ Snake}}\\\hline
            1. & CLK         & in  & wejście głównego \\\hline
            2. & Reset       & in  & logiczny iloczyn zegara głównego i zegara gry\\\hline
            3. & x $\#7$     & in  & główna współrzędna wyświetlanej kratki\\\hline
            4. & y $\#6$     & in  & główna współrzędna wyświetlanej kratki\\\hline
            8. & leftRight$\#3$& in  & wejście sterujące wężem \\\hline
            6. & Part $\#3$  & out & wyjście informujące o tym jaka część węża ma być wyświetlana \\\hline
            7. & push        & in  & wejście logiki odpowiadającej za rozszerzenie się węża po zjedzeniu punktu \\\hline
            8. & max\_size$\#13$& out & wyjście wyniku (długości węża) w czasie gry wartość jest równa 0 \\\hline
        \end{tabularx}
    \end{table}
        \newpage
    \begin{table}[!ht]
        \begin{tabularx}{\textwidth}{|l|c|c|X|}\hline
            \multicolumn{4}{|c|}{\textbf{Top $\rightarrow$ Map\_controler $\rightarrow$ Apple\_generator}}\\\hline
            1. & CLK         & in  & wejście głównego \\\hline
            2. & Reset       & in  & logiczny iloczyn zegara głównego i zegara gry\\\hline
            3. & CanDraw     & in  & wejście blokujące w przypadku, gdy dana kratka jest zajęta\\\hline
            3. & x $\#7$     & in  & główna współrzędna wyświetlanej kratki\\\hline
            4. & y $\#6$     & in  & główna współrzędna wyświetlanej kratki\\\hline
            5. & apple       & out & wyjście koloru punkt, w momencie gdy VGA, zapyta o współrzędną $xy$, sygnał ustawiana jest na wysoki\\\hline
            6. & done        & out & wyjście informujące o tym, że punkt jest generowany i ma ustawione współrzędne\\\hline
            \hline
            \multicolumn{4}{|c|}{\textbf{Top $\rightarrow$ Map\_controler $\rightarrow$ Print\_digit}}\\\hline
            1. & CLK         & in  & wejście głównego \\\hline
            2. & Reset       & in  & logiczny iloczyn zegara głównego i zegara gry\\\hline
            3. & x $\#7$     & in  & główna współrzędna wyświetlanej kratki\\\hline
            4. & y $\#6$     & in  & główna współrzędna wyświetlanej kratki\\\hline
            5. & Data $\#13$ & in  & wejście wyniku, podawane w momencie przegranej\\\hline
            6. & Draw        & out & wyjście koloru, w momencie gdy VGA poda współrzędne danego punktu, sygnał draw wystawi informację czy kratka ma być zapełniona czy nie\\\hline
        \end{tabularx}
        \caption{Zawartość modułu}
    \end{table}
    \subsection{Dodatkowe pliki}
    \subsection{Pamięci węża i mapy}
        \tab Wąż wyposażony jest w dwie pamięci:
        \begin{enumerate}
            \item 1 bitowa pamięć RAM -- o wymiarach równych wymiarom mapy, w której przechowywane są poszczególne segmenty węża,
            \item 2 bitowy rejestr FIFO -- o głębokości równej iloczynowi wymiarów mapy ($\text{deeph} = x\cdot y$).
        \end{enumerate}
        Oba te moduły zostały wygenerowane przez środowisko jako IP Core.
    \subsubsection{Generator pseudolosowy}
        \tab Generator liczb pseudolosowych zastosowany w projekcie to prosty 16 bitowy scrambler.
        Rozwiązania równań zostały wzięte z dokumentu PDF, przygotowanego przez Xilinx'a.
        Rozdzielczość scramblera pozwala na dość dokładną emulację wartości losowej.

        16 bitowa rozdzielczość, pozwala na pokrycie dużo większej przestrzeni niż cała mapa.
        Dlatego, wartości odczytywane z scramblera, zostały podzielone w taki sposób, że:
        \begin{align}
            \text{starsza połówka}\ modulo(\text{szerokość mapy}) \Rightarrow x \\
            \text{młodsza połówka}\ modulo(\text{wysokość mapy}) \Rightarrow y 
        \end{align}
    \newpage
    \subsubsection{Wyświetlanie wyniku}
        \tab Wynik odczytywany odczytywany jest z licznika wbudowanego w rejestr FIFO
        i przekazywany do modułów zamieniających Naturalny Kod Binarny $(NKB)$ w kod $BCD_{10}$, zgodnie z wzorem:
        \begin{align}
            &\text{Ones} &=& &\frac{Data}{1}\ modulo(10) \\
            &\text{Tens} &=& &\frac{Data}{10}\ modulo(10) \\
            &\text{Hundreds} &=& &\frac{Data}{100}\ modulo(10) \\
            &\text{Thousands} &=& &\frac{Data}{1000}\ modulo(10)
        \end{align}

        Następnie z uprzednio zaprogramowanej pamięci ROM (również stworzonej jako IP Core),
        odczytywane są wartości, poszczególnych kratek w zależności dla każdej liczby, a ich wartość (kolor wyświetlany) przekazywana jest do układu sterującego \textit{VGA}.
    % \subsubsection{}

