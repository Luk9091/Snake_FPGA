\subsection{Dodatkowe pliki}
    \subsection{Pamięci węża i mapy}
        \tab Wąż wyposażony jest w dwie pamięci:
        \begin{enumerate}
            \item 1 bitowa pamięć RAM -- o wymiarach równych wymiarom mapy, w której przechowywane są poszczególne segmenty węża,
            \item 2 bitowy rejestr FIFO -- o głębokości równej iloczynowi wymiarów mapy ($\text{deeph} = x\cdot y$).
        \end{enumerate}
        Oba te moduły zostały wygenerowane przez środowisko jako IP Core.
    \subsubsection{Generator pseudolosowy}
        \tab Generator liczb pseudolosowych zastosowany w projekcie to prosty 16 bitowy scrambler.
        Rozwiązania równań zostały wzięte z dokumentu PDF, przygotowanego przez Xilinx'a.
        Rozdzielczość scramblera pozwala na dość dokładną emulację wartości losowej.

        16 bitowa rozdzielczość, pozwala na pokrycie dużo większej przestrzeni niż cała mapa.
        Dlatego, wartości odczytywane z scramblera, zostały podzielone w taki sposób, że:
        \begin{align}
            \text{starsza połówka}\ modulo(\text{szerokość mapy}) \Rightarrow x \\
            \text{młodsza połówka}\ modulo(\text{wysokość mapy}) \Rightarrow y 
        \end{align}
    \newpage
    \subsubsection{Wyświetlanie wyniku}
        \tab Wynik odczytywany odczytywany jest z licznika wbudowanego w rejestr FIFO
        i przekazywany do modułów zamieniających Naturalny Kod Binarny $(NKB)$ w kod $BCD_{10}$, zgodnie z wzorem:
        \begin{align}
            &\text{Ones} &=& &\frac{Data}{1}\ modulo(10) \\
            &\text{Tens} &=& &\frac{Data}{10}\ modulo(10) \\
            &\text{Hundreds} &=& &\frac{Data}{100}\ modulo(10) \\
            &\text{Thousands} &=& &\frac{Data}{1000}\ modulo(10)
        \end{align}

        Następnie z uprzednio zaprogramowanej pamięci ROM (również stworzonej jako IP Core),
        odczytywane są wartości, poszczególnych kratek w zależności dla każdej liczby, a ich wartość (kolor wyświetlany) przekazywana jest do układu sterującego \textit{VGA}.
    % \subsubsection{}
