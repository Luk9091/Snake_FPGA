\section{Podsumowanie}
    \tab Powyższy projekt był bardzo przyjemnym doświadczeniem, 
    pokazującym olbrzymią przewagę układów FPGA nad mikroprocesorami
    w sytuacjach, kiedy trzeba generować przebiegi o stałych parametrach,
    niezależnie od pozostałych operacji wykonywanych przez układ.
    Przykładowo generowanie sygnałów VGA, które na mikrokontlorerze jest w większości niemożliwe (lub bardzo trudne) a na bramkach logicznych jest to bajecznie proste.\\
    Z drugiej strony projekt pokazał też wady programowalnych układów logicznych i ich słabości w wykonywaniu
    wydawałoby się prostych (ale bardzo wyspecjalizowanych) operacji, takich jak dodawanie czy liczenie reszty z dzielenia.

    \subsection*{Wniosek końcowy}
        \tab Układy FPGA, są niezwykle wygodną formą budowania dużych 
        i skomplikowanych układów logicznych. 
        Jednak, współpraca z zewnętrznymi wyspecjalizowanymi układami 
        (t.j.: m.in.: pamięci RAM, mikroprocesory czy układy ALU) 
        jest niezbędna do optymalizacji zużycia zasobów.
