\section{Opis funkcjonalny}
    % zdjęcie układu
    \begin{figure}[!ht]
        \centering
        \includegraphics[width=0.4\textwidth]{mimas.jpg}
        \renewcommand{\figurename}{Zdjęcie}
        \caption{Mimas -- płytka rozwojowa z układem Spartan 6}
        \label{fig:mimas}
    \end{figure}
    Na powyższym zdjęciu \ref{fig:mimas} przedstawiono zdjęcie płytki na której przeprowadzono testy.
    Głownym interfejsem komunikacyjnym są cztery przyciski podpisane prze producenta (licząc od dołu) SW1 do Sw4.
    \begin{enumerate}
        \item [SW4 -- ] reset wewnętrzny,
        \item [SW3 -- ] przycisk sterowania - skręć w lewo,
        \item [SW2 -- ] przycisk sterowania - skręć w prawo,
        \item [SW1 -- ] reset główny.
    \end{enumerate}
% 
    Nad przyciskami znajduje się pasek ośmiu ledów, które podczas gry wyświetlają animację sygnalizującą
    pracę układu.

    \subsection{Opis rozgrywki}
        \tab Zasady gry są proste gracz jest głową żarłocznego węża 
        i jego zadaniem jest tak kierować swoimi ruchami aby zjeść jak najwięcej czerwonych punktów.
        Jednocześnie starając się nie uderzyć w żadną ze ścian planszy oraz nie zjeść samego siebie.
        Dodatkowym utrudnieniem jest rozszerzanie się węża za każdym razem gdy zostanie zjedzony punkt.
        % Rozszerzanie to odbywa się poprzez ruc

        % \newpage
        Zachowanie węża przedstawia prosta maszyna stanów:
        \begin{figure}[!ht]
            \centering
            \begin{tikzpicture}
                \draw
                    ( 0,  0) node[draw, circle, align=center, minimum width = 2.5cm](start){Sprawdzenie\\kratki}
                    ( 4, -2) node[draw, circle, align=center, minimum width = 2.5cm](head){Ruch\\głowy}
                    ( 4, -6) node[draw, circle, align=center, minimum width = 2.5cm](capture){Zajęcie\\kratki}
                    ( 0, -8) node[draw, circle, align=center, minimum width = 2.5cm](tail){Ruch\\ogona}
                    (-4, -6) node[draw, circle, align=center, minimum width = 2.5cm](free){Zwolnienie\\kratki}
                    (-4, -2) node[draw, circle, align=center, minimum width = 2.5cm](wait){Czekaj}

                    ( 6,  0) node[draw, circle, align=center, minimum width = 1cm](reset){Reset}
                ;
                \draw [thick, -Straight Barb] (start)    edge[bend left] (head);
                \draw [thick, -Straight Barb] (head)     edge[bend left] (capture);
                \draw [thick, -Straight Barb] (capture)  edge[bend left] (tail);
                \draw [thick, -Straight Barb] (capture)  edge[bend left] node[rotate = -25, above]{Dodaj punkt} (wait);
                \draw [thick, -Straight Barb] (tail)     edge[bend left] (free);
                \draw [thick, -Straight Barb] (free)     edge[bend left] (wait);
                \draw [thick, -Straight Barb] (wait)     edge[bend left] (start);
                % \draw [thick, -Straight Barb] (wait)     edge[loop left] (wait);
                \draw [thick, -Straight Barb] (reset)     edge[bend right] (head);
                \draw [thick, -Straight Barb] (start)    edge[bend left] node[rotate = 90, above]{Zniszcz się} (tail);
                \draw [thick, -Straight Barb] (free)     edge[bend left] node[rotate =-25, above]{Zniszcz się} (tail);
            \end{tikzpicture}

            \renewcommand{\figurename}{Schemat}
            \caption{Algorytm działania węża}
            \label{schematic:snake}
        \end{figure}
        
        Powyższy opis (schemat \ref*{schematic:snake}) należy rozbudowa o kilka słów komentarza,
        \begin{enumerate}
            \item Sprawdzenie kratki, sprawdza stan sprawdza nie tylko czy nie wystąpiła kolizja, ale także czy został zdobyty punkt. 
            Na tej podstawie wystawiany jest specjalny, odpowiadający za dodanie punkt lub za samozniszczenie.
            \item W momencie ruchu głowy na rejestrze FIFO, zapisywany jest kierunek ruchu, który po odczytaniu służy ogonowi za pamięć w jakim kierunku ma się poruszyć.
            \item W trakcie samozniszczenia, powstaje pętla między, która traw do restartu procesu. Jednak ogon nie porusza się w nieskończoność a jedynie do momentu opróżnienia rejestru FIFO.
            \item Zajęcie/zwolnienie kratki w rzeczywistości odbywa się w tym samym takcie zegarowym co ruch głowy/ogona, jednak ze względu na czytelność te operacje zostały wyróżnione.
        \end{enumerate}

        
    \subsection{Generowanie punktów}
        \tab Miejsce pojawienia się punktu na mapie jest 16 pseudolosową liczbą, z której górna połówka odpowiada za współrzędną $y$, natomiast dolna za współrzędną $x$.
        Tak wygenerowane współrzędne następnie są porównywane z mapą i jeśli kratka o tych współrzędnych jest wolna, pojawia się punkt.

    \subsection{Przegrana}
        \tab Gracz kończy rozgrywkę w momencie zderzenia z samym sobą lub ze ścianą.
        W chwili zderzenia, wyzwolone zostają sygnały sterujące animacją destrukcji
        a wynik gracza zostaje zatrzaśnięty oraz wyświetlony na planszy.


